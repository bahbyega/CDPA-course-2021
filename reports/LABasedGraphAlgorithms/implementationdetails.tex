\section{Детали реализации}
Каждый из алгоритмов оперирует над вершинами и ребрами графа, которые имеют различное представление в упомянутых 
библиотеках. В pygraphblas и SciPy граф представлен в разреженном матричном виде. Networkx использует вложенные 
словари.

Алгоритмы получают на вход граф в соответствующем реализации представлении. Также для поиска в ширину и алгоритма 
Беллмана-Форда передается начальная вершина, от которой происходит отсчет.

Выходными данными являются, в случае поиска ширину, --- вектор посещенных вершин, значениями которого являются 
уровни обхода графа, на которых находится соответствующая вершина. В случае подсчета треугольников --- общее 
количество треугольников (циклов длины 3) в графе, в случае алгоритма Беллмана-Форда --- вектор кратчайших 
расстояний от начальной вершины до всех остальных (достижимых) вершин.

Поговорим подробнее о реализации каждого алгоритма.

Реализация на pygraphblas использует GraphBLAS API, предоставляющий элементы линейной алгебры для построения 
графовых алгоритмов.

Так, поиск в ширину построен на векторно-матричном умножении. Умножая вектор \emph{v} булевых значений 
текущих вершин в обходе, на матрицу смежности \emph{A} графа мы можем получить булев вектор, в котором на 
\emph{i}-м месте будет храниться информация о достижимости \emph{i}-й вершины от уже найденных вершин. 
Умножение \emph{v} на \emph{A}$^2$ даст вектор достижимых вершин на расстоянии двух переходов от текущих 
вершин и так далее. Векторно-матричное умножение при этом происходит в булевом полукольце, где операции 
поэлементного сложения и умножения работают со значениями True и False. Остается лишь на каждом шаге присваивать 
значение счетчика уровня обхода тем вершинам, что были помечены достижимыми.

Подсчет треугольников основан на умножении матриц. В наивной реализации можно трижды перемножить матрицу смежности 
на саму себя и получить количество циклов длины 3 для каждой вершины на соответствующей строчке главной диагонали. 
После чего сложить элементы главной диагонали и поделить сумму на 6, чтобы получить общее число треугольников в 
графе. Однако такой подход неэффективен, так как с каждым умножением матрица становится плотнее и ее дальнейшее 
умножение на саму себя начинает занимать большее число операций. Реализованная версия получает сначала нижнюю 
треугольную часть матрицы, а затем умножает ее на саму себя, при этом применяя маску, то есть считаются элементы 
только в тех i-х и j-х ячейках, что были в первоначальной треугольной матрице. После чего полученные значения 
складываются, и эта сумма является общим числом треугольников в графе.

Алгоритм Беллмана-Форда так же, как и поиск в ширину основан на векторно-матричном умножении. Главное отличие 
здесь --- использование мин-плюс полукольца, в котором классические операции сложения и умножения заменяются 
на операции взятия минимума и сложения соответственно. 

Подробнее эти алгоритмы приведены в псевдокоде в листингах~\ref{bfs_pygrb},\ref{bford_pygrb},\ref{tri_pygrb}.

Алгоритмы поиска в ширину и подсчета треугольников на SciPy были реализованы на тех же принципах линейной алгебры. 
Использование булевых полуколец было заменено на сравнение полученных значений с нулем. В
листингах~\ref{bfs_SciPy},\ref{tri_SciPy} приводится код основных частей алгоритмов, которые синтаксически 
отличаются от кода, использующего pygraphblas.

Для Беллман-Форда на SciPy вызывается библиотечная функция (листинг~\ref{bford_SciPy}) из модуля scipy.sparse.csgraph 
(compressed sparse graph routines). В ней \emph{n - 1} раз расчитывается минимальная дистанция до каждой вершины, 
где \emph{n} --- число вершин в графе. После чего проверяется наличие отрицательных циклов в графе с помощью еще 
одной итерации по всем вершинам графа.

Также библиотечные функции вызываются для алгоритмов, использующих NetworkX 
(листинги~\ref{bfs_std},\ref{tri_std},\ref{bford_std}). Поиск в ширину в NetworkX реализован стандартным образом. 
В то время как Беллман-Форд использует улучшенную версию алгоритма под названием SPFA (Shortest Path Faster Algorithm), 
в которой, в отличие от стандартной реализации, используется очередь для хранения вершин, расстояние до которых 
уже было пересчитано. Вершины из этой очереди просматриваются первыми. В подсчете треугольников в алгоритме 
NetworkX каждый треугольник считается три раза, по одному разу в каждой вершине. Алгоритм возвращает словарь, 
сопоставляющий вершине количество треугольников в ней, после чего значения суммируются и делятся на три для 
получения общего числа треугольников в графе.

\vspace{10px}

\begin{minipage}{0.46\textwidth}
\begin{algorithm}[H]
\centering
\caption{(pygraphblas)\\Поиск в ширину. Псевдокод.}\label{bfs_pygrb}
\begin{verbatim}
# Input: A - adj matrix NxN
#        s - source vertex
# Output: v

v = [0, ... , 0]
q = [False, ... , False]
q[s] = True
level = 1

while level <= N and q
    v<q> = level  # mask q
    q = [False, ... , False]
    q<v> = q x A  # lor-land sem.
    level++
\end{verbatim}
\end{algorithm}
\end{minipage}\hfill
\begin{minipage}{0.46\textwidth}
\begin{algorithm}[H]
\centering
\caption{(pygraphblas)\\Беллман-форд. Псевдокод.}\label{bford_pygrb}
\begin{verbatim}
# Input: A - adj matrix NxN
#        s - source vertex
# Output: v

# check if graph is weighted

v = [inf, ... , inf]
v[s] = 0

for k = 0 to N-1:
    v = v min.+ A 
    
    # break if v not changing
    
\end{verbatim}
\end{algorithm}
\end{minipage}\par
\begin{minipage}{0.96\textwidth}
\begin{algorithm}[H]
    \centering
    \caption{(pygraphblas) Подсчет треугольников.}\label{tri_pygrb}
    \begin{verbatim}
    # Input: A - adj matrix
    # Output: r
    
    # check if graph is undirected
    
    # Sandia algorithm
    L = tril(A)
    R = L x L  # using mask L
    r = sum(R)
    \end{verbatim}
\end{algorithm}
\end{minipage}\hfill

\begin{minipage}{0.46\textwidth}
\begin{algorithm}[H]
\centering
\caption{(SciPy)\\Поиск в ширину. Основная часть.}\label{bfs_SciPy}
\begin{verbatim}
# initialize vects ...
not_empty = True; level = 1
while not_empty and\
level <= n_verts:
    for i in range(n_verts):
        if (found_nodes_vect[i]):
            res_vect[i] = level

    found_nodes_vect =\ 
        ((res_vect @ graph > 0)\
        - res_vect) > 0

    not_empty =\ 
        found_nodes_vect\ 
            .sum() > 0
    level += 1
# ... 
\end{verbatim}
\end{algorithm}
\end{minipage}\hfill
\begin{minipage}{0.46\textwidth}
\begin{algorithm}[H]
\centering
\caption{(SciPy)\\Подсчет треугольников.Основная часть.}\label{tri_SciPy}
\begin{verbatim}
# load lower portion 
# of adj matrix as
# adj_matrix_part ...

def triangular_adj_matr_count\
(adj_matrix_part):
    res_matr = adj_matrix_part\
        .multiply(adj_matrix_part\
        *adj_matrix_part)
        return int(res_matr.sum())
# ...
    
    
    
    


\end{verbatim}
\end{algorithm}
\end{minipage}

\begin{minipage}{0.96\textwidth}
\begin{algorithm}[H]
    \centering
    \caption{(SciPy) Беллман-Форд.}\label{bford_SciPy}
    \begin{verbatim}
    # from scipy.sparse import csgraph
    def sp_bellman_ford(graph, src_vertex):
        return csgraph.bellman_ford(graph, indices=src_vertex,
                                return_predecessors=False)
    \end{verbatim}
\end{algorithm}
\end{minipage}\hfill

\begin{minipage}{0.96\textwidth}
\begin{algorithm}[H]
    \centering
    \caption{(NetworkX) Поиск в ширину, стандатная реализация.}\label{bfs_std}
    \begin{verbatim}
    # import networkx as nx
    def std_bfs(graph, src_vertex):
        res = nx.single_source_shortest_path_length(graph, src_vertex)
        return [dist+1 for _, dist in sorted(res.items())]
    \end{verbatim}
\end{algorithm}
\end{minipage}\hfill

\begin{minipage}{0.96\textwidth}
\begin{algorithm}[H]
    \centering
    \caption{(NetworkX) Подсчет треугольников, стандатная реализация.}\label{tri_std}
    \begin{verbatim}
    def std_triangles_count(graph):
        if nx.is_directed(graph):
            raise Exception("Graph is not undirected")
        return sum(nx.triangles(graph).values()) // 3
    \end{verbatim}
\end{algorithm}
\end{minipage}\hfill

\begin{minipage}{0.96\textwidth}
\begin{algorithm}[H]
    \centering
    \caption{(NetworkX) Беллман-Форд стандатная реализация.}\label{bford_std}
    \begin{verbatim}
    def std_bellman_ford(graph, src_vertex):
        res = nx.single_source_bellman_ford_path_length(graph, src_vertex)
        return [dist for _, dist in sorted(res.items())]
    \end{verbatim}
\end{algorithm}
\end{minipage}\hfill