\section{Заключение}

В результате проведенных экспериментов было получено:

\begin{enumerate}
    \item Реализация алгоритмов с помощью операций линейной алгебры на pygraphblas оказалось самой эффективной. 
    Наибольшая разница обнаружилась в алгоритме подсчета треугольников. Это можно объяснить тем, что его реализация 
    в большей степени основана на перемножении матриц, которое в pygraphblas максимально оптимизировано.
    В поиске в ширину были отмечены наименьшие различия. Это можно обосновать использованием меньшего числа операций 
    линейной алгебры в реализации.
    
    \item Хочется отметить, что в реализации алгоритмов поиска в ширину и подсчета треугольников на SciPy были использованы 
    операции линейной алгебры, из-за чего код реализации этих алгоритмов на pygraphblas и SciPy получился практически идентичным. 
    Из этого можно сделать вывод о том, что эти операции не так эффективно адаптированы в SciPy по сравнению с pygraphblas. 
    Тем не менее, с увеличением числа вершин графа реализации с помощью SciPy все заметнее опережали стандартные решения, 
    используемые NetworkX. 
    
    \item Беллман-Форд на SciPy использовал одноименную функцию из библиотеки\cite{csgraph}, что может объяснить такой 
    непропорционально большой отрыв во времени исполнения в сравнении с другими алгоритмами. По всей видимости, проверки 
    на отрицательные циклы повлияли на время исполнения алгоритма на больших графах. Однако результат оказался гораздо 
    медленнее ожидаемого, даже с учетом проверок.
    
    \item По полученным графикам на наборах данных \textit{amazon} можно судить о пропорциональной зависимости размеров 
    графов ко времени исполнения алгоритмов. Однако окончательный анализ о существовании такой зависимости стоит провести 
    на более разнородных данных. Возможно, стоит использовать датасеты графов с большей разницей в размерах, при этом имеющих 
    одинаковую структуру.
\end{enumerate}