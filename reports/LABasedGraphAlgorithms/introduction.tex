\section{Введение}

Современные компьютерные архитектуры позволяют легко обрабатывать линейные и иерархические структуры данных, 
такие как листы, стеки или деревья. Задачи обработки различных графов же зачастую имеют неструктурированный характер. 
В них отсутствует векторизация, в связи с чем распараллеливание и оптимизация алгоритмов на графах становятся трудными задачами, 
нерегулярный доступ к памяти вызывает промахи в кэше. В то же время алгоритмы на графах можно преобразовать к последовательности 
матрично-векторных операций, адаптировав для этого только базовые операции линейной алгебры. Что вместе со стандартизацией модели 
хранения различных видов графов в памяти в виде разреженной матрицы поможет упростить оптимизацию кода обработки графа.

В данной работе будет проведен анализ производительности алгоритмов на графах с использование
операций линейной алгебры. Автором были реализованы следующие алгоритмы: поиск в ширину, подсчет
треугольников, поиск кратчайших путей (алгоритм Беллмана-Форда). А именно, будет проведено
сравнение реализаций вышеперечисленных алгоритмов с помощью библиотеки pygraphblas, являющейся оберткой написанной на языке python 
над API спецификацией GraphBlas, предоставляющей набор стандартных операций над матрицами и векторами. Реализации алгоритмов с 
помощью библиотеки SciPy, предназначенной для решения различных научных и инженерных математических проблем, а также реализации, 
предоставляемой стандартной библиотекой анализа графов NetworkX, специально предназначенной для работы с графами и другими сетевыми 
структурами.


