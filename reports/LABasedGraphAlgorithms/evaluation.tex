\section{Проведение эксперимента}

Измерения производились на компьютере со следующими характеристиками: процессор AMD A10-5757M 2.5 GHz, 8 Гб оперативной памяти 
DDR3, под управлением операционной системы Ubuntu 20.04.2 LTS.

В качестве исходных данных были использованы датасеты SNAP (Stanford Network Analysis Platform\cite{snap}) взятые из SuiteSparse 
Matrix Collection --- коллекции разреженных матриц реальных данных\cite{suitesparse-research}. А именно, наборы данных 
ca-AstroPh\cite{suitesparse-data}, ca-CondMat, ca-HepTh описывающие соавторство в научных работах в виде неориентированного графа. 
Наборы amazon-0302, amazon-0312, amazon-0505, amazon-0601, представляющие собой ориентированные графы, собранные парсингом сайта 
Amazon с промежутком в несколько месяцев, а также неориентированный граф социальной сети com-Youtube, в котором более миллиона вершин.

Эксперимент был поставлен следующим образом. В память программы загружался граф из датасета, после чего случайным образом выбиралась 
начальная вершина для поиска в ширину и поиска кратчайшего пути (для реализованного алгоритма подсчета треугольников в графе начальная 
вершина не требуется). Затем к этому графу последовательно применялись упомянутые алгоритмы и измерялось время исполнения каждого. 
Для измерения времени использовалась библиотека time, значения сохранялись в долях секунды. Для датасетов \textit{ca} выполнялось 
10 итераций, для остальных датасетов алгоритмы исполнялись 5 раз ввиду больших размеров графов. Полученные временные значения 
записывались в файл.

На рисунках 1-4 представлены результаты проведенных измерений.

\begin{figure}[htp]
\centering
    \begin{subfigure}{0.5\textwidth}
        \includegraphics[width=\textwidth]{images/ca-HepTh.pdf}
        \subcaption{Набор данных ca-HepTh}
    \end{subfigure}\hfill
    \begin{subfigure}{0.5\textwidth}
        \includegraphics[width=\textwidth]{images/ca-AstroPh.pdf}
        \subcaption{Набор данных ca-AstroPh}
    \end{subfigure}\par
    \begin{subfigure}{0.5\textwidth}
        \includegraphics[width=\textwidth]{images/ca-CondMat.pdf}
        \subcaption{Набор данных ca-CondMat}
    \end{subfigure}\hfill
\caption{Времена работы алгоритмов на наборах данных \textit{ca}.\\Среднее 10 замеров.}
\end{figure}

Рисунок 1 иллюстрирует среднее время исполнения алгоритмов реализованных с помощью разных библиотек в виде гистограммы. 
Такое представление удобно использовать в случае, когда данные можно сгруппировать по категориям. На временной оси используется 
логарифмический масштаб. Это связано с тем, что алгоритм Беллмана-Форда в реализации на SciPy работает существенно медленнее 
остальных вариантов.

На графе другого типа --- amazon-0302, состоящем из большего числа вершин и ребер было проверено, не является ли плохая 
производительность алгоритма поиска кратчайшего пути на SciPy зависимой от входных данных первого эксперимента. Алгоритм 
поиска треугольников к графам типа \textit{amazon} на применялся, так как написанные реализации считают треугольники только 
в неориентированном графе. Результаты представлены на рисунке 2.

\begin{figure}[ht]
\begin{center}
\scalebox{0.85}{
    \includegraphics{images/amazon-0302.pdf}
}
\caption{\label{amazon-0302}Результаты работы алгоритмов на графе amazon-0302.\\Среднее 5 замеров.}
\end{center}
\end{figure}

После чего производительность реализаций поиска в ширину и алгоритма Беллмана-Форда были проанализированы на графах amazon-0312, 
amazon-0505, amazon-0601. Они интересны тем, что были собраны с одной сети с разницей не больше чем в месяц друг от друга, 
благодаря чему производительность алгоритмов можно оценить на одинаково структурированных начальных данных с разным количеством 
вершин и ребер (рисунок 3). Время исполнения Беллмана-Форда на SciPy было принято за 0, чтобы на графике линейного масштаба разница 
в производительности алгоритмов была видна нагляднее.

\begin{figure}[htp]
\centering
    \begin{subfigure}{0.5\textwidth}
        \includegraphics[width=\textwidth]{images/amazon-0302.1.pdf}
        \subcaption{Набор данных amazon-0302}
    \end{subfigure}\hfill
    \begin{subfigure}{0.5\textwidth}
        \includegraphics[width=\textwidth]{images/amazon-0312.pdf}
        \subcaption{Набор данных amazon-0312}
    \end{subfigure}\hfill
    \begin{subfigure}{0.5\textwidth}
        \includegraphics[width=\textwidth]{images/amazon-0505.pdf}
        \subcaption{Набор данных amazon-0505}
    \end{subfigure}\hfill
    \begin{subfigure}{0.5\textwidth}
        \includegraphics[width=\textwidth]{images/amazon-0601.pdf}
        \subcaption{Набор данных amazon-0601}
    \end{subfigure}\hfill
\caption{Времена работы алгоритмов на наборах данных \textit{amazon}.\\Среднее 5 замеров.}
\end{figure}

В заключении был проанализирован граф с существенно превосходящим числом вершин и примерно равным числом ребер относительно графов 
\textit{amazon}. Результаты представлены на рисунке 4.

\begin{figure}[ht]
\begin{center}
\scalebox{0.85}{
    \includegraphics{images/com-Youtube.pdf}
}
\caption{\label{com-Youtube}Результаты работы алгоритмов на графе com-Youtube.\\Среднее 5 замеров.}
\end{center}
\end{figure}








