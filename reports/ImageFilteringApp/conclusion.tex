\clearpage
\specialsection{Заключение}

В результате проведенных экспериментов было получено:

\begin{enumerate}
    \item Реализация алгоритма на видеокарте оказалось самой эффективной для большинства входных данных в проведенных экспериментах. 
    Выяснилось, что при небольших размерах фильтра и самого изображения реализация на процессоре с многопоточностью является не менее быстрой. 
    По всей видимости, это происходит из-за траснспортировки данных между памятью процессора и видеокарты. Временные затраты на перемещение 
    данных в этом случае нивелируют эффект от более быстрых вычислений на ядрах видеокарты.
    
    \item Важно отметить, что алгоритм на видеокарте показывает меньшую зависимость от размеров входных данных. Так, на первом рисунке видно, 
    что при увеличении размеров изображения, время исполнения растет не так сильно, как в реализациях на процессоре. На втором рисунке можно 
    отметить, что при увеличении размеров фильтра и сохранении размеров изображения эта тенденция также сохраняется.
    
    \item Алгоритм, распаралеленный на процессоре при помощи OpenMP показал производительность в четыре раза выше однопоточного варианта, что кратно числу ядер 
    процессора. Это говорит об эффективной реализации многопоточности в этом стандарте. Однако, не смотря на это, при больших входных данных 
    эта реализация алгоритма существенно медленнее варианта на GPU.
    
\end{enumerate}

Таким образом, был проведен экспериментальный анализ, на основе которого можно сделать вывод о том, что для независимых вычислений матричных 
операций видеокарта более производительна, чем центральный процессор.
\clearpage