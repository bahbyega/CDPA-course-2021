\specialsection{Введение}

Применение фильтров позволяет накладывать различные эффекты на изображения.
При этом при увеличении размеров фильтра и изображения начинают возникать вычислительные сложности,
исследование которых позволяет оценить производительность различных вычислительных устройств в 
контексте примитивных операций линейной алгебры.

В качестве домашней работы автором было реализовано приложение с графическим интерфейсом позволяющее 
применять фильтры к изображениям формата BMP. В нем присутствуют следующие функции: просмотр изображения 
и информации о нем, задание матричного фильтра, применение фильтра как один раз, так и несколько раз 
последовательно, отображение изображения до и после применения фильтра, а также возможность сохранить 
результат и применить фильтр сразу к нескольким изображениям, хранящимся в отдельной папке.
Помимо этого, в реализации предусматривается возможность применять фильтры используя как центральный 
процессор, так и видеокарту.

В данной работе будет проведен анализ производительности и масштабируемости реализованного приложения. 
А именно, оценена возможность добавления новой функциональности в разработанный продукт, а также оценена 
скорость применения матричных фильтров с использованием различных вычислительных устройств. 

\pagebreak
