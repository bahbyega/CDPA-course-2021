\section{Проведение эксперимента}

Измерения производились на компьютере со следующими характеристиками: видеокарта AMD Radeon 8650G, процессор AMD A10-5757M 2.5 GHz, 
8 Гб оперативной памяти DDR3 под управлением операционной системы Ubuntu 20.04.3 LTS.

В качестве исходных данных были использованы изображения BMP разных размеров, взятых с сайта FileSamples\cite{filesamlpes}. 
А именно, четыре изображения с размерами 640×426, 1280×853, 1920×1280 и 5184×3456 пикселей соответственно, с размерами файлов --- 
798 Кб, 3.12 Мб, 7.03 Мб и 51.26 Мб. При этом все эти изображения являются одной и той же картинкой разных размеров.

Эксперимент был поставлен следующим образом. Пользователь запускал графический интерфейс приложения, выбирал картинку соответствующего 
размера, после чего в поле количество итераций фильтра выставлял значение 10 и применял фильтры с размерами матриц 5×5 и 9×9 соответственно. 
Перед этим в код программы в функции применения фильтра были внесены изменения, проводящие замеры на однопоточном процессоре, процессоре, 
использующем многопоточность, и видеокарте. Для измерения времени исполнения использовался заголовочный файл time.h стандартной библиотеки 
языка Си, измерялось реальное время, а не процессорное время вычисления, так как в параллельных программах значение процессорного времени 
не отражает того, как долго исполнялась программа. Замеры записывались в секундах.

Таким образом, применялось 10 итераций наложения фильтров с размерами 5×5 и 9×9 для каждого изображения исходных данных. 
Замерялось время работы алгоритма без учета времени загрузки изображения в память. Полученные временные значения записывались в файл. 
На рисунке 1 представлены результаты проведенных измерений.

\begin{figure}[htp]
\centering
    \begin{subfigure}{0.5\textwidth}
        \includegraphics[width=\textwidth]{images/5x5.pdf}
        \subcaption{Время исполнения на ядре 5x5}
    \end{subfigure}\hfill
    \begin{subfigure}{0.5\textwidth}
        \includegraphics[width=\textwidth]{images/9x9.pdf}
        \subcaption{Время исполнения на ядре 9x9}
    \end{subfigure}\hfill
\caption{Результаты измерений первого эксперимента.\\ Среднее 10 замеров.}
\end{figure}

Рисунок 1 иллюстрирует среднее время наложения фильтра в различных реализациях в виде гистограммы. Такое представление удобно использовать, 
в случае когда данные можно сгруппировать по категориям. В данном случае категориями являются различные версии алгоритма. На временной 
оси используется логарифмический масштаб. Это связано с тем, что на однопоточном процессоре алгоритм исполняется сильно дольше, чем в 
остальных реализациях.

Далее было решено оценить влияние размера фильтра на производительность алгоритма с одинаковыми исходными данными изображения. Для 
этого была выбрана одна картинка с размером 1280×853, добавлены новые фильтры с размерами 3×3 и 15×15, и тем же образом запущен эксперимент 
на 10 итерации с каждым фильтром для данной картинки. Результаты представлены на рисунке 2.

\begin{figure}[ht]
\begin{center}
\scalebox{0.85}{
    \includegraphics{images/1280x853.pdf}
}
\caption{\label{com-Youtube}Результаты измерений второго эксперимента.\\Среднее 10 замеров.}
\end{center}
\end{figure}








